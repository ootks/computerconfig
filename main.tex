\documentclass[a4paper]{article}
\usepackage[margin=.8in]{geometry}
\usepackage{graphicx}
\usepackage{fancyhdr, multicol}
\usepackage{float}
\usepackage[all,cmtip]{xy}
\usepackage[linesnumbered,ruled,vlined]{algorithm2e}
\fancyhead[L]{\LARGE{Kevin Shu}}
\fancyhead[C] {\LARGE{Set 1}}
\fancyhead[R]{\LARGE{Ma130b}}

\author{Kevin Shu}
\title{Set 1}

%Set up fancy headers.
\pagestyle{fancy}
\usepackage{amsmath, amsthm, amsfonts, amssymb}
\usepackage[parfill]{parskip}

%Define theorem formatting
\newtheorem{theorem}{Theorem}
\newtheorem{lemma}{Lemma}

\newtheoremstyle{named}{}{}{\itshape}{}{\bfseries}{.}{.5em}{\thmnote{#3's }#1}
\theoremstyle{named}
\newtheorem*{namedtheorem}{Theorem}

%Make sections actually useful
\let\tempa\section
\renewcommand{\section}[1]{\tempa*{#1}}
\let\tempb\subsection
\renewcommand{\subsection}[1]{\tempb*{#1}}
\let\tempc\subsubsection
\renewcommand{\subsubsection}[1]{\tempc*{#1}}
\newcommand{\answer}[1]{{\subsubsection{Answer: #1}}}
\newcommand{\R}{\mathbb{R}}
\newcommand{\N}{\mathbb{N}}
\newcommand{\Q}{\mathbb{Q}}
\newcommand{\E}{\mathbb{E}}
\newcommand{\PP}{\textbf{P}}
\newcommand{\SPACE}{\textbf{SPACE}}
\newcommand{\NP}{\textbf{NP}}
\newcommand{\SAT}{\textbf{SAT}}
\newcommand{\pard}[2]{\frac{\partial #1}{\partial #2}}
\DeclareMathOperator*{\argmin}{arg\,min}
\DeclareMathOperator*{\argmax}{arg\,max}
\newcommand{\st}{{\text{ s.t. }}}

\begin{document}
\maketitle
\newpage
\section{3}
Consider the poset category $\mathbb{N}$, where there is a unique morphism from $n$ to $m$ if $n | m$. We see that this is a filtered category and indeed a directed set, since between any two elements $m$ and $n$, there is a glb, given by the LCM of the two integers.

We can consider $\frac{1}{n}\mathbb{Z}$, and a morphism $\phi :\frac{1}{n}\mathbb{Z}\rightarrow\frac{1}{m}\mathbb{Z}$ if $n | m$, by sending $\phi(\frac{a}{n}) = \frac{ka}{m}$, where $kn = m$. We see that this is clearly a group homomorphism and that this set of morphisms is an inductive system.

Now, we note that there is a morphism $\phi_n : \frac{1}{n}\mathbb{Z} \rightarrow \mathbb{Q}$, given by sending $\phi(\frac{a}{n}) = \frac{a}{n}$. We see that this commutes with all $\phi_n$, since $\frac{ka}{kn} = \frac{a}{n}$ in $\mathbb{Q}$. We also see that if $\{\gamma\}_n$ is any collection of maps from the $\frac{1}{n}\mathbb{Z}$ to some ring $A$ commuting with all $\phi_n$, then we can map $\mathbb{Q}$ into $A$ by sending $\frac{k}{n}$ to $\gamma_n(\frac{k}{n})$. This is a group homomorphism for reasons that we will detail in question 4. We see that this is the unique way of mapping $\mathbb{Q}$ to $A$ since any element of $\mathbb{Q}$, and therefore, $\mathbb{Q}$ is in fact the colimit pf this diagram.

More generally, we can define the localization of a ring $R$ with respect to a multiplicatively closed set as follows: we define an inductive system with elements in $S$ by having a unique morphism $\phi : s\rightarrow t$ for $s, t \in S$ if $\exists k \in S$ so that $ks = t$. We see that this is filtered because for any $s$, $t$, $st \ge s, t$.

We then consider $\frac{1}{s}R$ to be the ring given by the elements $\{\frac{r}{s}\}$, and note that we can include $\frac{1}{s}R$ into $\frac{1}{t}R$ for $t \ge s$ by sending $\frac{r}{s} \rightarrow \frac{kr}{t}$. 

We then have that the localization $R_S$ is the inductive limit of these $\frac{1}{s}R$. 
\section{4}
\subsection{a}
Let $\alpha : I \rightarrow Ring$ be a filtered inductive system of rings, i.e. a functor from a finite category to the category of rings. We can also think of $\alpha$ as an inductive system in the category of $Set$, which has an inductive limit, given by 
\[
    \lim_{\leftarrow i \in I^{o}} A_i = \bigcup A_i / \sim 
\]
where $\sim$ is the equivalence relation given by $a \sim b$ for $a \in A_i$ and $b \in A_j$ if there is some $f_{ik}$ and $f_{jk}$ so that $f_{ik}(a) = f_{jk}(b)$. We see that this is the equivalence relation generated by the equations $a = f_{ik}(a)$ for all $k \ge i$ and $a \in A_i$.

We then see that there are ring operations, where if $a \in A_i$ and $b \in A_j$, and $k \ge i, j$, then $a + b = f_{ik}(a) + f_{jk}(b)$, where the addition is performed in $A_k$. We note that this is well defined, since our inductive system guarantees that if $\ell \ge k \ge i, j$, then $f_{i\ell}(a) = f_{k\ell }(f_{ik}(a))$ and $f_{j\ell}(b) = f_{ k\ell }(f_{jk}(a))$, and our equivalence relation guarantees that $f_{i\ell}(a) + f_{j\ell}(b) = f_{k\ell}(f_{ik}(a) + f_{jk}(b)) = f_{ik}(a) + f_{jk}(b)$.

Similarly, we have that the corresponding definition for multiplication given by $ab = f_{ik}(a)f_{jk}(b)$ if $a \in A_i$ and $b \in A_j$, and $k \ge i, j$ is well defined. 


We have that  since the zero element in any ring is always mapped to the zero element under a ring homomorphism, that if we choose $0 \in A_i$ for any $i$, this will be a zero element in the colimit. Similarly, $1 \in A_i$ for any $A_i$ will be a 1 element in the colimit. 
 
We also have an additive inverse to any element $a \in A_i$, given by $-a \in A_i$. Also, addition is commutative, since it is commutative in each $A_i$.

Also, it is clear that multiplication and addition distribute appropriately. This shows that the colimit is a ring.
\subsection{b}
We have that if $f_{ij}: A_i \rightarrow A_j$ is a morphism, then any $A_j$ module is naturally a $A_i$ module. Hence, we have that an inductive limit is defined as follows:

We define an inductive system of modules  over a directed set $I$ as a set of $A$ modules $M_{i}$, and a collection of functions $f_{i,j} : M_i \rightarrow M_j$ for $i \le j$, so that for $i \le j \le k$, $f_{i,k} = f_{i,j}\circ f_{j,k}$, and each $f_{i,j}$ is an $A_j$-module homomorphism.

We then have that the inductive limit is defined as follows: the underlying set is
\[
    \lim_{\leftarrow i \in I^{o}} M_i = \bigcup M_i / \sim
\]
where $\sim$ is the equivalence relation given by $m \sim n$ for $m \in M_i$ and $n \in M_j$ if there is some $f_{ik}$ and $f_{jk}$ so that $f_{ik}(a) = f_{jk}(b)$.

We can then define addition as follows: for $m \in M_i$ and $n \in M_j$, and $k \ge i,j$,
\[
    m + n = f_{ik}(m) + f_{jk}(n)
\]
This is well defined for the same reasons as in the case of rings: we have that if we chose some $\ell \ge k$ to perform this addition in, we would have that 
\[
    m + n = f_{i\ell}(m) + f_{j\ell}(n) = f_{k\ell}(f_{ik}(m) + f_{jk}(n))
\]

We see that since module homomorphisms always map the 0 element to the 0 element, there is a zero element for addition, given by the 0 element in any $M_i$.

We also have an additive inverse to any element $m \in M_i$, given $-m \in M_i$.

We also have an $A$-action on $\lim_{\leftarrow i \in I^{o}} M_i$. Let $a \in A$ and $m \in M_j$, then we can in fact take $a \in R_k$, where $k \ge j$. We can then let
\[
    a m = a f_{jk}(m)
\]
We see that this is well defined, since if $\ell \ge k \ge j$, then we can find 
\[
    a m = a f_{j\ell}(m) = f_{k\ell}(a)f_{k\ell}(f_{jk}(m)) =  f_{k\ell}(af_{jk}(m))
\]
where we have used the fact that $f_{k\ell}$ is an $A_{\ell}$ module homomorphism, and the $A_k$ action on $A_{\ell}$ is given by $a m = f_{k\ell}(a)m$.

This shows that the $A$-action is well defined, and we have that similarly, since $1$ in any $A_i$ acts as the identity, and multiplication distributes for each $A_i$ map, it distributes here as well. Therefore,  we have that

\[
    \lim_{\leftarrow i \in I^{o}} M_i = \bigcup M_i / \sim
\]
\subsection{c}
We define a module homomorphism
\[
    f : \lim_{i \in I} N_{i} \otimes_{A_i} M_i \rightarrow 
    \lim_{i \in I} N_{i} \otimes_A \lim_{i \in I} M_i
\]

We note that we can represent an element of
\[
\lim_{i \in I} N_{i} \otimes_{A_i} M_i
\]
by an element $a = \sum_{\ell } n_{\ell} \otimes m_{\ell} \in N_{i} \otimes_{A_i} M_i$ for some $i$.

We then map this to the element $\sum_{\ell } n_{\ell} \otimes m_{\ell} \in \lim_{i \in I} N_{i} \otimes_A \lim_{i \in I} M_i$. 

We see that this map is well defined, since we have that if $b = \sum_{\ell } n'_{\ell} \otimes m'_{\ell} \in N_{k} \otimes_{A_k} M_k$ is another representative in the same class as $a$, where $k \ge i$, then there is some map $\phi_{jk} \otimes \psi_{jk}$ so that $(\phi_{jk} \otimes \psi_{jk})(a) = b$, and then we have that
\[
   (\phi_{jk} \otimes \psi_{jk})(f(a)) = (\phi_{jk} \otimes \psi_{jk}) \sum_{\ell } n_{\ell} \otimes m_{\ell}
   =\sum_{\ell } \phi_{jk} (n_{\ell}) \otimes \psi_{jk}(m_{\ell} )
\]

We also see that this is clearly an $A$ module homormophism.

We also note that this is an isomorphism, since we see that it is surjective, since any element in $\lim_{i \in I} N_{i} \otimes_A \lim_{i \in I} M_i$ can be represented as $\sum_{\ell } n_{\ell} \otimes m_{\ell} \in N_{i} \otimes_{A_i} M_i$, where we can take both terms in the tensor factors to be in $M_i \otimes N_i$ by choosing $i$ large enough.


Now, to show that the kernel is zero, suppose that $f(a) \sim 0$, then $f(a)$ must map to 0 under some $\phi_{j}\otimes \phi_k$, which implies that $a$ must map to 0, and therefore, $a \sim 0$, as desired.

\section{5}
\subsection{a}
Let $F$ be the forgetful functor from $Top$ to $Set$.

Consider the functor $Disc:Set \rightarrow Top$ sending a set $X$ to the topological space on $X$ with the discrete topology, and any function of sets to the corresponding function on topological spaces. We see that the corresponding function is in fact continuous, since the inverse image of any set in $X$ is open in the discrete topology.

For $X \in Set$ and $Y \in Top$, we can identify $Hom_{Set}(X, F(Y))$, the functions from $X$ to the underlying set of $Y$, with $Hom_{Top}(Disc(X), Y)$, the set of continuous functions from $Disc(X)$ to $Y$, since any function from $Disc(X)$ to $Y$ is continuous.

Therefore, $Disc$ is a left adjoint to $F$.

Consider the functor $Triv : Set \rightarrow Top$ sending a set $X$ to the topological space on $X$ with the trivial topology, and any function of sets to the corresponding function on topological spaces. We see that the corresponding function is in fact continuous, since any function into a set with the triviail topology is continuous.


For $X \in Set$ and $Y \in Top$, we can identify $Hom_{Set}(F(Y), X)$, the functions from $F(Y)$ to $X$, with $Hom_{Top}(Y, Triv(X))$, the set of continuous functions from $Y$ to $Triv(X)$, since any function from $F(Y)$ to $X$ is continuous when $X$ has the trivial topology.

\subsection{b}
Since $Top$ has right and left adjoint functors into $Set$, we have that we can identify the inductive and projective limits in $Top$ with those in $Set$.

\[
    \lim_{\rightarrow i \in I}X = \{x_i \in \sqcup_I X_i :x_i \sim x_j \text{ if } \exists f_{ij}, x_j = f_{ij}(x_i)\}
\]
with the map from $X_i$ to $\lim_{\leftarrow \in I}X_i$ given by the corresponding function guarenteed by the colimit property of this set, together with the quotient topology generated by these functions.
\[
    \lim_{\leftarrow \in I}X_i= \{(x_i)_{i\in I} \in \prod_I X_i : \forall f_{ij}, x_j = f_{ij}(x_i)\}
\]
with the map from $\lim_{\leftarrow \in I}X_i$ to $X_i$ given by the corresponding function guarenteed by the limit property of this set, together with the weakest topology that makes all of those maps continuous.

\subsection{c}
Since injective and surjective functions are monomorphisms and epimorphisms respectively in the category of $Sets$, so morphisms in  $Top$ are in particular functions, this implies that injective and surjective morphisms in $Top$ will also be monomorphisms and epimorphisms respectively.

Now, it also suffices to check that a monomorphism in the category of $Top$ must also be a monomorphism in the category of $Set$. To show this, note that if $f:X\rightarrow Y$ is a monomorphism in the category of $Top$, and let $g$ and $h$ be any functions from $Y$ to $Z$. If we give $Z$ the trivial topology, then $g$ and $h$ are continuous. Therefore, if $g \circ f = h \circ f$, then $g = h$, and so $f$ is a monomorphism in the category of $Set$, which implies that $f$ must be injective.

Similarly, using the discrete topology functor, we can show that any epimorphism in the category of $Top$ is surjective, as desired.

\subsection{d}
A continuous function $\phi$ with another morphism $\psi$ so that $\phi \circ \psi = Id$ and $\psi \circ \phi = Id$ is by definition a homeomorphism.

On the other hand, we note that the identity function from the Sierpinski space to $\{0,1\}$ with the trivial topology is continuous, injective and surjective, but its inverse is not continuous.

\section{6}
We first note that $I_u^Y$ is filtered, since for any pair of objects $(X, g), (X', g')$, we can consider their coproduct: $(X \sqcup X', g\sqcup g')$, and note that there is a map from both $(X, g)$ and $(X', g')$ to $(X \sqcup X', g\sqcup g')$. 

Therefore, in the category of sets, the colimit $u_!(F)(X)$ exists. We now need to define what $u_{!}(F)$ does to morphisms. If $\gamma : A \rightarrow B$ is a morphism in $\mathcal{C}'$, then we note that we obtain a functor $\phi   :I_u^{B} \rightarrow I_u^A$, so that
\[
    \phi(X, g) = (X, g \circ f)
\]
and any morphism $f : (X, g)\rightarrow (X', g')$ is sent to $f:(X, g \circ f)\rightarrow (X', g' \circ f)$. 

Now, we note that there are morphisms $F(X) \in I$ to $u_!(F)(B)$. We further note that these morphisms commute with the diagram defining $u_!(F)(A)$, so there is a unique morphism $u_!(F)(\gamma)$ commuting with this diagram from $u_!(F)(A)$ to $u_!(F)(B)$.

Now, we must show that $u_!(F)$ is in fact a functor. We see that this is true because if we have morphisms $\gamma : A \rightarrow B$ and $\delta : B \rightarrow C$, then because $u_!(\gamma \circ \delta)$ was defined in terms of a universal property, it suffices to show that $u_!(\gamma( \circ u_!(\delta)$ also satisfies that universal property.

Now, we must show adjointness, so that 
\[
Hom_{\hat{C}'}(u_!(F), G) \cong Hom_{\hat{C}}(F, u^*(G))
\]
in a functorial manner. 

We begin by considering a morphism $\phi: u_!(F) \rightarrow G$, which has the form of a natural transformation with components at each $Y$ given by $\phi_Y$. 

A map $\phi_Y$ from $u_!(F)(Y)$ to $G(Y)$ is canonically equivalent to a collection of maps from each $F(Y)$ to $G(Y)$ which commutes with all of the morphisms in $I_u^X$. Given such a collection of maps for each $Y \in \hat{\mathcal{C}'}$, we wish to construct $\psi_X$, a component of a natural transformation from $F(X)$ to $G(u(X))$.

Now, we note that if $Y = u(X)$, then there is an element of $I_u^Y$ given by $(X, id)$. We then have a morphism $\delta_X : F(X) \rightarrow u_!(F)(Y)$, from the colimit property of $u_!(F)(Y)$. We wish to show that $\delta_X$ is a natural transformation, and $\psi_X$ will just be the horizontal composition of natural transformations $\delta$ and $\phi$.


We then have that we can compose this with $\phi_Y$ to obtain a map from $F(X)$ to $G(u(X))$. We now claim that $\psi_X = \phi_Y \circ \delta$ is the desired component of a natural transformation.

To show this, let $f : X \rightarrow A$ be a morphism in $\hat{\mathcal{C}}$. We wish to show that $F(f)\circ \psi_X = \psi_A \circ G(u(f))$.
We will do this using the natural transformation property of $\phi$

\subsection{b}
We can construct the right adjoint of $u^*$ by dualizing all of the arrows: consider the category $J_u^Y$ of pairs $(X, g)$ where $g$ is a map from $g:u(X)\rightarrow Y$. We then define 
\[
    u_*(F)(X) = \lim_{\leftarrow (X, g) \in J_u^Y} \mathcal{F}(X)
\]


\end{document}
